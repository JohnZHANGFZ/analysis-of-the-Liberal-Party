% Options for packages loaded elsewhere
\PassOptionsToPackage{unicode}{hyperref}
\PassOptionsToPackage{hyphens}{url}
%
\documentclass[
  ignorenonframetext,
]{beamer}
\usepackage{pgfpages}
\setbeamertemplate{caption}[numbered]
\setbeamertemplate{caption label separator}{: }
\setbeamercolor{caption name}{fg=normal text.fg}
\beamertemplatenavigationsymbolsempty
% Prevent slide breaks in the middle of a paragraph
\widowpenalties 1 10000
\raggedbottom
\setbeamertemplate{part page}{
  \centering
  \begin{beamercolorbox}[sep=16pt,center]{part title}
    \usebeamerfont{part title}\insertpart\par
  \end{beamercolorbox}
}
\setbeamertemplate{section page}{
  \centering
  \begin{beamercolorbox}[sep=12pt,center]{part title}
    \usebeamerfont{section title}\insertsection\par
  \end{beamercolorbox}
}
\setbeamertemplate{subsection page}{
  \centering
  \begin{beamercolorbox}[sep=8pt,center]{part title}
    \usebeamerfont{subsection title}\insertsubsection\par
  \end{beamercolorbox}
}
\AtBeginPart{
  \frame{\partpage}
}
\AtBeginSection{
  \ifbibliography
  \else
    \frame{\sectionpage}
  \fi
}
\AtBeginSubsection{
  \frame{\subsectionpage}
}
\usepackage{amsmath,amssymb}
\usepackage{lmodern}
\usepackage{iftex}
\ifPDFTeX
  \usepackage[T1]{fontenc}
  \usepackage[utf8]{inputenc}
  \usepackage{textcomp} % provide euro and other symbols
\else % if luatex or xetex
  \usepackage{unicode-math}
  \defaultfontfeatures{Scale=MatchLowercase}
  \defaultfontfeatures[\rmfamily]{Ligatures=TeX,Scale=1}
\fi
\usetheme[]{Pittsburgh}
\usecolortheme{orchid}
\usefonttheme{structurebold}
% Use upquote if available, for straight quotes in verbatim environments
\IfFileExists{upquote.sty}{\usepackage{upquote}}{}
\IfFileExists{microtype.sty}{% use microtype if available
  \usepackage[]{microtype}
  \UseMicrotypeSet[protrusion]{basicmath} % disable protrusion for tt fonts
}{}
\makeatletter
\@ifundefined{KOMAClassName}{% if non-KOMA class
  \IfFileExists{parskip.sty}{%
    \usepackage{parskip}
  }{% else
    \setlength{\parindent}{0pt}
    \setlength{\parskip}{6pt plus 2pt minus 1pt}}
}{% if KOMA class
  \KOMAoptions{parskip=half}}
\makeatother
\usepackage{xcolor}
\newif\ifbibliography
\usepackage{longtable,booktabs,array}
\usepackage{calc} % for calculating minipage widths
\usepackage{caption}
% Make caption package work with longtable
\makeatletter
\def\fnum@table{\tablename~\thetable}
\makeatother
\usepackage{graphicx}
\makeatletter
\def\maxwidth{\ifdim\Gin@nat@width>\linewidth\linewidth\else\Gin@nat@width\fi}
\def\maxheight{\ifdim\Gin@nat@height>\textheight\textheight\else\Gin@nat@height\fi}
\makeatother
% Scale images if necessary, so that they will not overflow the page
% margins by default, and it is still possible to overwrite the defaults
% using explicit options in \includegraphics[width, height, ...]{}
\setkeys{Gin}{width=\maxwidth,height=\maxheight,keepaspectratio}
% Set default figure placement to htbp
\makeatletter
\def\fps@figure{htbp}
\makeatother
\setlength{\emergencystretch}{3em} % prevent overfull lines
\providecommand{\tightlist}{%
  \setlength{\itemsep}{0pt}\setlength{\parskip}{0pt}}
\setcounter{secnumdepth}{-\maxdimen} % remove section numbering
\ifLuaTeX
  \usepackage{selnolig}  % disable illegal ligatures
\fi
\IfFileExists{bookmark.sty}{\usepackage{bookmark}}{\usepackage{hyperref}}
\IfFileExists{xurl.sty}{\usepackage{xurl}}{} % add URL line breaks if available
\urlstyle{same} % disable monospaced font for URLs
\hypersetup{
  pdftitle={Liberal Party Analysis Report},
  pdfauthor={Chloe Zhang, Yijing Shen, Fangzheng Zhang, TianHuiqi Chen},
  hidelinks,
  pdfcreator={LaTeX via pandoc}}

\title{Liberal Party Analysis Report}
\subtitle{Suggestions to Liberal Party and Justin Trudeau}
\author{Chloe Zhang, Yijing Shen, Fangzheng Zhang, TianHuiqi Chen}
\date{December 7, 2020}

\begin{document}
\frame{\titlepage}

\begin{frame}{Overall Introduction}
\protect\hypertarget{overall-introduction}{}
We would like to suggest one of the Canadian federal political parties -
the Liberal Party.\\
In order to give suggestions to the leader and the Party, we are going
to discover and get conclusions from voters' preferences:\\
1. We will estimate a range of voters' age who will vote for the Liberal
Party to understand which age group is the dominating supporter of the
Liberal Party.\\
2. We will build a linear regression model to discover the association
between the score of how voters feel about Justin Trudeau as the leader
of Liberal Party, and the score of how citizens feel about the Liberal
Party in general, in which to discover whether voters consider Justin
Trudeau as a qualified Prime Minister Candidate and a Liberal Party
leader.\\
3. We will estimate a range of rating on Justin Trudeau to discover if
citizens do not satisfy about Liberal Party because of Justin Trudeau
not keeping the election promises he made in 2015.\\
\end{frame}

\begin{frame}{Data Summary}
\protect\hypertarget{data-summary}{}
We use different filtered data subsets from the original data set
`ces19' in each research question.\\

\begin{itemize}
\item
  In the first research study, we filter out the people whose first vote
  choice is Liberal Party and save the data as a subset named
  `ces\_liberal\_supporter'. Then, from the subset, we randomly select a
  sample with size of 2000 and save it as `observed\_data' to calculate
  the test statistic.\\
\item
  In the second research study, we filter out NA values in row
  `party\_rating\_23' and `lead\_rating\_23' and select only these party
  and leader rating variables from the filtered data and save it as the
  population named with `model2'.\\
\item
  In the third research study, we take out only the group fo people who
  were not very satisfied about the Liberal Party from the
  `fed\_gov\_sat' variable, then select `fed\_gov\_sat',
  `lib\_promises', and `lead\_rating\_23' and save the data as the
  population for this study.
\end{itemize}
\end{frame}

\begin{frame}{Topic1: Age Range Estimation}
\protect\hypertarget{topic1-age-range-estimation}{}
\includegraphics{Group100Project_files/figure-beamer/unnamed-chunk-2-1.pdf}

We will discover the first question by using bootstrapping method, to
estimate a plausible range of voters' age whose first choice is the
Liberal Party, in order to gain information about the primary age group
of the Liberal Party supporters. Since the main focus of this research
question is to estimate a reasonable range of age of Liberal Party
supporters, bootstrapping is a appropriate method to be used which
creates a range of plausible values for the population parameter. With
information about the age group, we are able to focus more on propagate
the other age groups that are valid to vote in next election to attract
potential supporters.
\end{frame}

\begin{frame}{Analysis}
\protect\hypertarget{analysis}{}
\begin{longtable}[]{@{}ll@{}}
\toprule()
2.5\% & 97.5\% \\
\midrule()
\endhead
49.16650 & 50.66354 \\
\bottomrule()
\end{longtable}

The table above displays the result of our bootstrap sampling
estimation, which means that we are 95\% confident that the mean age of
citizens who choose the Liberal Party as first choice in 2019 is between
49.16650 years old and 50.66354 years old. In bootstrap sampling, we
rapidly random-pick samples of size 2000 without replacement from the
initial sample (also with size 2000) picked from the population, which
in this research topic is people whose first choice on election is
Liberal party in 2019. In each bootstrap sample, we calculate the
average age and record the genenral distribution.
\end{frame}

\begin{frame}{Suggestions}
\protect\hypertarget{suggestions}{}
Since most Liberal Party supporters in 2019 were in the age range from
49 to 51, we consider that young adults and seniors are less intersted
in political events comparing with middle-aged adults. Therefore, we
suggest to propagate the Liberal party to young adults and seniors more
often than the middle aged group while also keeping the attention from
people in the middle age range to Liberal Party to maintain their
interests and supports.
\end{frame}

\begin{frame}{Topic 2: Party and Leader Rating}
\protect\hypertarget{topic-2-party-and-leader-rating}{}
\includegraphics{Group100Project_files/figure-beamer/unnamed-chunk-3-1.pdf}

In the second research topic, we will discover the linear association
between party rating of Liberal Party and leader rating of Justin
Trudeau in order to gain insights about Canadian's feeling about the
leader of Liberal Party. Since we want to study the relationship between
two variables, we use a single linear regression model. We set up the
linear model based on the null hypothesis that the slope of the fitted
linear model is equal to 0. \[H_0: \beta_1 = 0\] \[H_1: \beta_1 \neq 0\]
\end{frame}

\begin{frame}{Analysis}
\protect\hypertarget{analysis-1}{}
The table below displays coefficients of the linear regression:

\begin{longtable}[]{@{}lllll@{}}
\toprule()
& Estimate & Std. Error & t value &
Pr(\textgreater\textbar t\textbar) \\
\midrule()
\endhead
(Intercept) & -0.41015 & 0.13946 & -2.94090 & 0.00327 \\
party\_rating\_23 & 0.94117 & 0.00237 & 396.67596 & 0.00000 \\
\bottomrule()
\end{longtable}

From the table, the p-value of the hypothesis testing is 0 means we have
very strong evidence to reject the null hypothesis that the slope of the
fitted linear model equals to 0. Thus, combining with information from
the scatter plot, we can visually summarize that there are positive
linear relationship between the Liberal party rating and the leader
rating. Since the correlation is 0.907 which is close to 1, the two
rating variables have strong linear association. The value of
correlation(r) and coefficient of determination(R\^{}2) are
representative data that implies the test the reliability and persuasion
of our model.\\
\end{frame}

\begin{frame}{Analysis}
\protect\hypertarget{analysis-2}{}
Correlation is a value between -1 and 1, which is used to show the
strength and direction of the linear relationship between voters'
feeling of Liberal Party and Justin Trudeau. The value we got for
correlation is 0.907, which means for a big part of voters, their
feeling of Justin Trudeau is increasing with their feeling of Liberal
Party.\\
Coefficient of determination is a value between 0 and 1 which is used to
show how much variability in y is explained by the fitted regression
line. The value we got for coefficient of determination is 0.82, which
means our fitted regression line is very persuasive to show the
variability of how the change of voters' feeling about Liberal Party
influence voters' feeling of Justin Trudeau.\\
The slope of the fitted linear model is 0.94 which means every 1 score
increase on party rating of Liberal results 0.94 higher rating on Justin
Trudeau.
\end{frame}

\begin{frame}{Suggestions}
\protect\hypertarget{suggestions-1}{}
From the result of linear regression model, we may figure out the
information that Justin Trudeau political performances are based on the
ideology of Liberal Party. Therefore, Justin Trudeau is a suitable
candidate of Prime Minister of Canada to represent the Liberal Party as
the leader. Additionally, if Justin Trudeau continues his political
ideology under Liberal Party's philosophy, the party can maintain a
relatively stable supporting rate.
\end{frame}

\begin{frame}{Topic 3: Leader Rating Estimation}
\protect\hypertarget{topic-3-leader-rating-estimation}{}
\includegraphics{Group100Project_files/figure-beamer/unnamed-chunk-4-1.pdf}
\end{frame}

\begin{frame}{Analysis}
\protect\hypertarget{analysis-3}{}
In the third research topic, we will estimate the average leader rating
of Justin Trudeau from people that are not quite satisfied about Liberal
Party with the reason of Justin Trudeau not keeping his promises in the
2015 election. Since we expect a estimated result to be a possible range
of the actual value of the average rating of Justin Trudeau,
bootstrapping sampling is a efficient method to apply. The population
for this research question is people who answered the survey in 2019 and
responded they are not very satisfied about the federal government at
the time, which was lead by the Liberal Party. We focus on the main
cause of dissatisfaction on how the population thought on whether Justin
Trudeau kept his promises from 2015 election. From the bar plot, we may
summarize that most of the population disagree that Justin Trudeau had
kept his promises, and it is a causation of dissatisfaction on Liberal
Party. Therefore, we are reasonable to expect the average rating of
Justin Trudeau from the population is relatively low.
\end{frame}

\begin{frame}{Result and Suggestion}
\protect\hypertarget{result-and-suggestion}{}
\begin{longtable}[]{@{}lllll@{}}
\toprule()
2.5\% & 25\% & 50\% & 75\% & 97.5\% \\
\midrule()
\endhead
37.20086 & 37.80725 & 38.13575 & 38.48113 & 39.08401 \\
\bottomrule()
\end{longtable}

From the summary table of result above, we are 95\% confident that the
mean rating of Justin Trudeau from people who are not very satisfied
about Liberal Party in 2019 is between 37.80725 to 39.08401 out of 100.~

By the data we shown above, we can see most citizens who gave low marks
on Justin Trudeau since he broke his promise from 2015 election. Thus,
we suggest Justin Trudeau as the leader of Liberal Party try to keep his
promises made during the election in his capacity, in order to earn
trust and preference from the population. Or alternatively, he may want
to make amounts of less promises, instead, to make exquisite, realistic
promises.
\end{frame}

\begin{frame}{Limitations}
\protect\hypertarget{limitations}{}
\begin{enumerate}
\tightlist
\item
  The data set was from 2019, therefore, it is not up\_to\_date for
  using on research study in 2020.\\
\item
  The limitation of topic 1 and topic 3 is the reliability of our result
  is rely on the reliability of the original sample. Since All data
  shown in the graph is calculated from the original sample, we must
  ensure it is well selected from the original data set. Also, if there
  are extreme values in the original data set and being taken in the
  original sample, then bootstrapping model may underestimate the
  variability of the original data set.\\
\end{enumerate}
\end{frame}

\begin{frame}{Limitations}
\protect\hypertarget{limitations-1}{}
\begin{enumerate}
\setcounter{enumi}{2}
\tightlist
\item
  There may exist confounding variables in both topic 2 and topic 3. In
  topic 2, the party rating variable is not the only variable that
  affect the leader rating while they may also affect the party rating,
  such as feeling on immigration and LGBT minorities. In the topic 3,
  the causation of dissatisfaction toward the Liberal Party can be other
  than feelings on Justin Trudeau, for example, immigration policies may
  also be the reason of people unsatisfied about the party.\\
\item
  Since topic 2 involves with linear regression model, the existence of
  residual (extreme values) may affect the accuracy and coverage of the
  fitted model.\\
\end{enumerate}
\end{frame}

\begin{frame}{Overall Conclusion}
\protect\hypertarget{overall-conclusion}{}
In consequence, we suggest the Liberal Party pursue propaganda to the
groups of young adults and senior based on the result from research
topic 1;\\
From topic 2, the rating of the feeling of Justin Trudeau and Liberal
Party has a positive linear relationship. Therefore, we suggest Justin
Trudeau to continue his political ideology and carry forward the
policies of Liberal Party;\\
Combining the research result of topic 2 and 3, we may conclude that
although Justin Trudeau has good political concepts, if he cannot
fulfill public's expectations from his promises, he may lose supports
from the public.
\end{frame}

\end{document}
